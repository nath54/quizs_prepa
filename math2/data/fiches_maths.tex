\documentclass[11pt,twoside,a4paper]{article}

\usepackage[margin=1in]{geometry} 
\usepackage{amsmath}
\usepackage{tcolorbox}
\usepackage{amssymb}
\usepackage{amsthm}
\usepackage{lastpage}
\usepackage{fancyhdr}
\usepackage{accents}
\usepackage{enumitem}
\usepackage[french]{babel}
\pagestyle{fancy}
\setlength{\headheight}{40pt}


\title{Fiche de révisions - Maths}
\author{Cerisara Nathan}
\date{Juillet 2022}

\begin{document}

\maketitle

% Poly 1

\newpage
\tableofcontents


\newpage
\section{Logique et raisonnements}

\subsection{Rudiments de logique}

\subsection{Rédaction, Raisonnement, Démonstrations}

\newpage
\section{Ensembles}

\subsection{Théorie intuitive des ensembles}

\subsection{Paradoxes ensemblistes et axiomatisation}

\newpage
\section{Applications}

\subsection{Applications}

\subsection{Image directe, image réciproque}

\subsection{Injectivité, surjectivité, bijection}

\newpage
\section{Sommes et produits}

\subsection{Manipulation des signes $\sum$ et $\prod$}

\subsection{Sommes classiques à connaître}

\newpage
\section{Relations}

\subsection{Définitions générales}

\subsection{Relations d'équivalence}

\subsection{Relations d'ordre}

\newpage
\section{Des entiers naturels aux réels}

\subsection{Un mot sur $\mathbb{N}$ et $\mathbb{Z}$}

\subsection{Nombres rationnels}

\subsection{Nombres réels}

\subsection{Intervalles}

\subsection{Droite achevée $\overline{\mathbb{R}}$}

\newpage
\section{Nombres complexes}

\subsection{Définitions et manipulations}

\subsection{Trigonométrie}

\subsection{Racines d'un nombre complexe}

\subsection{Nombres complexes et géométrie}

\newpage
\section{Cardinaux et dénombrement}

\subsection{Cardinaux des ensembles finis}

\subsection{Combinatoire des ensembles d'applications}

\subsection{Combinatoire des sous-ensembles}

\subsection{Bijection, déesse de la combinatoire}

\subsection{Preuves combinatoires d'identités}

% Poly 2

\newpage
\section{Dérivation de fonctions}

\subsection{Rappel sur les limites}

\subsection{Dérivation}

\subsection{Fonctions convexes}

\subsection{Etude d'une fonction}

\newpage
\section{Les fonctions usuelles}

\subsection{Prérequis}

\subsection{Exponentielle, logarithme, puissances}

\subsection{Fonctions trigonométriques}

\subsection{Réciproques des fonctions trigonométriques}

\subsection{Fonctions hyperboliques}

\subsection{Réciproques des fonctions hyperboliques}

\subsection{Tableau des dérivées des fonctions usuelles}

\newpage
\section{Calcul intégral}

\subsection{Calcul intégral et primitivation}

\subsection{Techniques de calcul intégral}

\subsection{Rapide introduction aux intégrales impropres}

\newpage
\section{Équations différentielles linéaires}

\subsection{EDL}

\subsection{EDL d'ordre 1}

\subsection{EDL d'ordre 2 à coefficients constants}

\newpage
\section{Suites numériques}

\subsection{Convergence de suites}

\subsection{Propriétés des suites liées à la convergence}

\subsection{Suites extraites}

\subsection{Etudes de suites particulières}

\newpage
\section{Propriété des fonctions $\mathcal{C}^0$ ou $\mathcal{D}^1$ sur un intervalle}

\subsection{Fonctions continues sur un intervalle}

\subsection{Fonctions dérivables sur un intervalle}

\newpage
\section{Calcul asymptotique }

\subsection{Domination, négligeabilité}

\subsection{Equivalents}

\newpage
\section{Approximations polynomiales}

\subsection{Formule de Taylor-Young et DL usuels}

\subsection{Généralités sur les DL}

\subsection{Opérations sur les DL}

\subsection{Developpements asymptotiques}

\subsection{Applications}

\subsection{DL des fonctions à 2 variables}

\subsection{DL des fonctions usuelles}

\newpage
\section{Séries numériques}

\subsection{Notion de série et de convergence}

\subsection{Séries à termes positifs}

\subsection{Etude de la semi-convergence}

\subsection{Familles sommables}

\subsection{Autour de la série exponentielle}

\subsection{Dérivée des séries géométriques}

\newpage
\section{Intégration}

\subsection{Intégrale des fonctions en escalier}

\subsection{Construction de l'intégrale de Riemann}

\subsection{Primitives et intégration}

% Poly 3

\newpage
\section{Structures algébriques}

\subsection{Lois de composition}

\subsection{Structure}

\subsection{Groupes}

\subsection{Anneaux et corps}

\newpage
\section{Calcul matriciel}

\subsection{Opérations matricielles}

\subsection{Matrices carrées}

\subsection{Pivot de Gauss et matrices équivalentes par lignes}

\subsection{Résolution d'un système linéaire}

\subsection{Produit matriciel par bloc}

\newpage
\section{Arithmétique des entiers}

\subsection{Divisibilité, nombres premiers}

\subsection{PGCD, PPCM}

\subsection{Entiers premiers entre eux}

\subsection{Décomposition primaire d'un entier}

\subsection{Théorème des restes chinois}

\newpage
\section{Polynômes et fractions rationnelles}

\subsection{Polynômes à coefficients dans un anneau commutatif}

\subsection{Arithmétique dans $\mathbb{K}[X]$}

\subsection{Racines d'un polynôme}

\subsection{POlynômes irréductibles dans $\mathbb{C}[X]$ et $\mathbb{R}[X]$}

\subsection{Fractions rationnelles}

\subsection{Primitivations des fractions rationnelles réelles}

% Poly 4

\newpage
\section{Algèbre linéaire}

\subsection{Notion d'espace vectoriel}

\subsection{Familles de vecteurs}

\subsection{Applications linéaires}

\subsection{AL et familles de vecteurs}

\subsection{Sous-espaces affines d'un EV}

\newpage
\section{Dimension finie}

\subsection{EV de dimension finie}

\subsection{AL en dimension finie}

\subsection{AL et matrices}

\subsection{Changement de base}

\subsection{Formes linéaires et hyperplans}

\newpage
\section{Groupes symétriques}

\subsection{Notations de cycles}

\subsection{Signature d'une permutation}

\subsection{Décomposition cyclique d'une permutation}

\subsection{Cycles et signature}

\newpage
\section{Déterminants}

\subsection{Définition des déterminants}

\subsection{Calculs des déterminants}

\newpage
\section{Espaces préhilbertiens réels}

\subsection{Produits scalaires}

\subsection{Orthogonalité}

\subsection{Espaces euclidiens}

% Poly 5

\newpage
\section{Espaces probabilisés}

\subsection{Espaces probabilisables}


\noindent\fbox{

    \begin{minipage}{\textwidth}
    
    \textbf{Définition 28.1.1 (Univers)}
    
    \begin{itemize}
    
        \item[$\bullet$] Un \textit{résultat}, ou une \textit{issue} de l'expérience est une donnée issue de l'expérience aléatoire; une même expérience peut fournir différents résultats, suivant ce qu'on veut étudier de l'expérience. 
    
        \item[$\bullet$] L' \textit{univers des possibles} $\Omega$ (ou plus simplement l'\textit{univers}), est l'ensemble des issues possibles d'une expérience.
        Une même expérience peut fournir plusieurs univers différents suivant ce que l'on veut en tirer. 
    
    \end{itemize}
    
    \end{minipage}
    }

\bigskip


\noindent\fbox{

    \begin{minipage}{\textwidth}
    
    \textbf{Définition 28.1.2 (Evenement, définition intuitive)}
    
    Un évenement est un sous-ensemble de $\Omega$. Pour certaines raisons techniques, lorsque $\Omega$ n'est pas fini, on est parfois amené à se restreindre et à ne pas considérer tous les sous-ensembles comme des évenements, ce que nous formaliserons plus loin avec la notion de tribu
    
    \end{minipage}
    }

\subsection{Espaces probabilisés}

\subsection{Conditionnement et indépendance}

\subsection{Les trois théorèmes fondamentaux}

\subsection{Principes généraux du calculs des probabilités}

\newpage
\section{Variables aléatoires}

\subsection{Variables aléatoires}

\subsection{Espérance mathématique}

\subsection{Variance, dispersion, moments}

\subsection{Indépendance de variables aléatoires}

\subsection{Covariance}

\subsection{Lois discrètes classiques}

\subsection{Inégalités et convergence}


\end{document}
