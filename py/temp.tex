
\documentclass[11pt]{article}

\usepackage{sectsty}
\usepackage{graphicx}

% Margins
\topmargin=-0.45in
\evensidemargin=0in
\oddsidemargin=0in
\textwidth=6.5in
\textheight=9.0in
\headsep=0.25in

\title{ Title}
\author{ Author }
\date{\today}

\newcommand{\bb}[1]{\mathbb{#1}}
\newcommand{\cal}[1]{\mathcal{#1}}
\newcommand{\frak}[1]{\mathfrak{#1}}
\newcommand{\ol}[1]{\overline{#1}}
\newcommand{\N}{\bb{N}}
\newcommand{\Z}{\bb{Z}}
\newcommand{\D}{\bb{D}}
\newcommand{\Q}{\bb{Q}}
\newcommand{\C}{\bb{C}}
\newcommand{\R}{\bb{R}}
\newcommand{\cR}{\cal{R}}
\newcommand{\B}{\bb{B}}
\newcommand{\gt}{>}
\newcommand{\lt}{<}

\begin{document}



\title{ Exported data from nath54/quizs prepa}
\author{ nath54}
\date{\today}


 \subtitle{Combinatoire}

\textb{ Définition de la cardinalité selon Frege }
\textit{  }
On dit que deux ensembles $E$ et $F$ ont le même cardinal s'il existe une bijection de $E$ à $F$. On note $\text{Card}(E) = \text{Card}(F)$
\textb{ Ensemble fini }
\textit{ Soit $E$ un ensemble. }
On dit que $E$ est fini si et seulement s'il existe un entier $n$ et une surjection $f: [\![1,n]\!] \rightarrow E$, où de façon équivalente s'il existe une injection $g : E \rightarrow [\![1,n]\!]$.
\textb{ Sous-ensemble d'un ensemble fini }
\textit{ Soit $F$ un sous ensemble de $E$. }
Si $E$ est fini, alors $F$ aussi.
\textb{ Bijections de $[\![1, n]\!]$ et $[\![1, m]\!]$ }
\textit{  }
Tout sous ensemble $F$ de $[\![1, n]\!]$ peut être mis en bijection avec un ensemble $[\![1, m]\!]$
\textb{ Egalité provenant de la bijection $[\![1, n]\!] \rightarrow [\![1, m]\!]$ }
\textit{ Soit $n$ et $m$ deux entiers. }
S'il existe une bijection de $[\![1, n]\!]$ sur $[\![1, m]\!]$, alors $n = m$.
\textb{ Cardinal d'un ensemble fini }
\textit{ Soit $E$ un ensemble fini. }
Il existe un unique entier $n$ tel qu'il existe une bijection $f: [\![1,n]\!] \rightarrow E$. L'entier $n$ est appelé cardinal de $E$, et noté $|E|$, ou Card($E$).
\textb{ Cardinal d'une union disjointe }
\textit{ Soit $A$, $B$, $A_1$, ..., $A_n$ des ensembles finis. }
<ol><li>Si $A \cap B = \emptyset\ $, alors $|A \sqcup B| = |A| + |B|$</li> <li>Plus généralement, si pour tout $(i, j) \in [\![1, n]\!]^2$ tel que $i \neq j$, $A_i \cap A_j = \emptyset\ $, alors $|A_1 \sqcup ... \sqcup A_n| = |A_1| + ... + |A_n|$</li></ol>
\textb{ Cardinal d'un complémentaire }
\textit{  }
Si $a \subset B$, alors $|\text{C}_BA|$ = |B| - |A|
\textb{ Cardinal d'un sous-ensemble }
\textit{  }
Si $A \subset B$, alors $|A| \leqslant |B|$, avec égalité si et seulement si $A=B$.
\textb{ Cardinal d'une union quelconque }
\textit{ Soient $A$ et $B$ des ensembles finis. }
On a : $|A \cup B| = |A| + |B| - |A \cap B|$
\textb{ Formule du crible de Poincaré }
\textit{ Soient $A_1, ..., A_n$ des ensembles finis. }
On a : $$|A_1 \cup ... \cup A_n| = \sum_{k=1}^{n}( (-1)^{k-1} \sum_{1 \leqslant i_{1} \leqslant ... \leqslant i_{k} \leqslant n} |A_{i_1} \cap \ldots\! \cap A_{i_k}|)$$ <br/> $$ = \sum_{I \subset [\![1,n]\!], I \neq \emptyset\ }((-1)^{|I|-1}|\bigcap_{i \in I}(A_i)|)$$
\textb{ Cardinal d'un produit cartésien }
\textit{ Soient $A$, $B$, $A1,...,A_n$ des ensembles finis. }
<ol><li>$|A \times B| = |A| \times |B|$</li> <li>Plus généralement, $|A_1 \times ... \times A_n| = \Pi_{i=1}^n|A_i|$</li></ol>
\textb{ Cardinal et injectivité }
\textit{ Soient $E$ et $F$ deux ensembles finis, et soit $f: E \longrightarrow F$ une application }
Si $f$ est injective, $\text{Card}(E) \leqslant \text{Card}(F)$
\textb{ Cardinal et surjectivité }
\textit{ Soient $E$ et $F$ deux ensembles finis, et soit $f: E \longrightarrow F$ une application }
Si $f$ est surjective, $\text{Card}(E) \geqslant \text{Card}(F)$
\textb{ Cardinal et bijectivité }
\textit{ Soient $E$ et $F$ deux ensembles finis, et soit $f: E \longrightarrow F$ une application }
Si $f$ est bijective, $\text{Card}(E) = \text{Card}(F)$
\textb{ Caractérisation des bijections }
\textit{ Soient $A$ et $B$ deux ensembles finsi de même cardinal, et $f: A \longrightarrow B$. }
Les trois propriétés sont équivalentes : <ol><li>$f$ est bijective</li> <li>$f$ est injective</li> <li>$f$ est surjective</li> </ol>
\textb{ Cardinal de l'ensemble des applications }
\textit{ Soient $E$ et $F$ deux ensembles finis.<br/>On rappelle qu'on note $F^E$ l'ensemble des applications de $E$ vers $F$. }
$|F^E| = |E^F|$
\textb{ p-listes }
\textit{  }
Une p-liste d'éléments de $F$ (ou p-uplet) est un élément ($x_1$, ..., $x_p$) de $F^p$.
\textb{ Nombre de p-liste }
\textit{  }
Le nombre de p-listes d'éléments de $F$ est $|F|^p$
\textb{ Cardinal de l'ensemble des parties }
\textit{  }
$|\mathcal{P}(E)| = 2^{|E|}$
\textb{ Lemme du berger }
\textit{ Soit $f: E \longrightarrow F$ une application surjective. <br/> On suppose qu'il existe un entier $k \in \N^* $ tel que pour tout $y \in F$, $|f^{-1}(y)| = k$ (tous les éléments de $F$ ont le même nombre $k$ d'antécédents). }
$|E| = k \times |F|$
\textb{ Dénombrement des injections }
\textit{ Soit $A$ et $B$ deux ensembles de cardinaux respectifs $p$ et $n$. }
Alors si $p \leqslant n$, le nombre d'injections de $A$ vers $B$ est $A_n^p  = \frac{n!}{(n-p)!}$. <br/> Si $p \gt n$, il n'existe pas d'injection de $A$ vers $B$.
\textb{ Dénombrement des p-arrangements }
\textit{ Soit $F$ de cardinal $n$ et $p \leqslant n$. }
Le nombre de p-listes d'éléments distincts de $F$ (ou p-arrangements de $F$) est $A_n^p = \frac{n!}{(n-p)!}$
\textb{ Nombre de permutations d'un ensemble }
\textit{ $\mathfrak{S}E$ représente l'ensemble des permutations de $E$ }
<ol> <li>Soit $E$ un ensemble fini. Alors $|\mathfrak{S} E| = |E|!$</li> <li>En particulier $|\mathfrak{S}_n| = n!$</li></ol>
\textb{ Coefficient binomial }
\textit{  }
Le coefficient binomial $\binom{n}{k}$ est le nombre de parties à $k$ éléments de $[\![1,n]\!]$
\textb{ Sens général du coefficient binomial }
\textit{  }
Le coefficient binomial $\binom{n}{k}$ est plus généralement le nombre de sous-ensemble de cardinal $k$ de n'importe quel ensemble $E$ de cardinal $n$.
\textb{ Expression factorielle du coefficient binomial }
\textit{  }
Pour $k \in [\![0, n]\!]$, $\binom{n}{k} = \frac{n!}{k!(n-k)!}$
\textb{ Propriétés du coefficient binomial }
\textit{ Soit $(n,k) \in \Z^2$. }
<ol> <li>$\binom{n}{k} = \binom{n}{n-k}$ (symétrie)</li> <li>$k\binom{n}{k} = n\binom{n-1}{k-1}$ (formule du comité-président)</li> <li>si $(n,k) \neq (-1, -1)$, $\binom{n}{k} + \binom{n}{k+1} = \binom{n+1}{k+1}$</li> </ol>
\textb{ Formule du binôme }
\textit{  }
Pour $n \in \N$, $$(a+b)^n = \sum_{k=0}^n\binom{n}{k}a^kb^{n-k}$$
\textb{ Principe fondamental du dénombrement }
\textit{  }
Pour montrer que deux ensembles ont le même cardinal, il suffit de construire une bijection entre eux. Ainsi, pour déterminer le cardinal d'un ensemble, on le met souvent en bijection avec un ensemble "de référence" dont on connaît le cardinal.
\textb{ Démonstration combinatoire d'une formule }
\textit{  }
<ol><li>Trouver un modèle adapté de la formule, autrement dit un ensemble d'objets dont le dénombrement fournira un des membres de l'égalité. Pour cela, il est préférable de s'aider du membre le plus simple de l'égalité</li> <li>Dénombrer cet ensemble de deux façons différentes. Souvent, on procède d'une part à un dénombrement direct, et d'autre part à un dénombrement après avoir effectué le tri (de façon formelle, cela revient à définir une partition de l'ensemble). Le résultat d'un dénombrement par tri se traduit par une somme.</li> <li>Evidemment, cette méhode n'est adaptée qu'à des formules portant sur des nombres entiers, si possible positifs. Il est parfois possible de se ramener à cette situation par un prétraitement de la formule à démontrer.</li></ol>
\textb{ Formule de Pascal (coefficients binomiaux) }
\textit{  }
$$\binom{n}{k} = \binom{n-1}{k-1} + \binom{n-1}{k}$$
\textb{ $$\sum_{k=0}^n \binom{n}{k}$$ }
\textit{  }
$$\sum_{k=0}^n \binom{n}{k} = 2^n$$
\textb{ Formule de Vandermonde (coefficients binomiaux) }
\textit{  }
$$\sum_{k=0}^n\binom{N}{k}\binom{M}{n-k} = \binom{N+M}{n}$$
\textb{ Formule de sommation sur une colonne }
\textit{  }
$$\sum_{k=0}^p\binom{n+k}{n} = \binom{n+p+1}{n+1}$$
\textb{ $$\sum_{k=0}^n = \binom{k}{N}\binom{n-k}{M}$$ }
\textit{  }
$$\sum_{k=0}^n = \binom{k}{N}\binom{n-k}{M} = \binom{n+1}{M+N+1}$$
\textb{ simplifier un $(-1)^n$ associé à un coefficient binomial }
\textit{  }
Remarquez qu'un signe $(-1)^n$ associé à un coefficient binomial correspond souvent à une comparaison de parités de cardinaux. On peut passer d'un cardinal pair à un cardinal impair, et vice-versa, en "allumant ou éteignant" un élément fixé à l'avance suivant qu'il est déjà ou non dans notre ensemble (plus précisement, il s'agit de l'opération $X \rightarrow X \Delta \{x\}$.)


 \subtitle{Limites - incomplet}

\textb{ Adhérence }
\textit{  }
Si $X$ est un sous(ensemble quelconque de $\R$, on peut considérer la limite en un point $a$ de l'adhérence $\ol{X}$ de $X$, défini comme étant le plus petit fermé contenant $X$, ou de façon équivalente, l'ensemble des points $x$ pouvant être approché d'aussi près qu'on veut par des points de $X$ (i.e. tout voisinage de $x$ rencontre $X$)
\textb{ Limite réelle lorsque $x$ tend vers $a$ }
\textit{ Soit $a \in \ol{X} \cap \R$ }
Soit $b \in \R$. On dit que $f(x)$ tend vers $b$ lorsque $x$ tend vers $a$ si :    $\forall \varepsilon \gt 0, \exists \eta \gt 0, \forall x \in X, |x − a| \leqslant \eta \implies |f(x)−b| \leqslant \varepsilon$
\textb{ Limite $+\infty$ lorsque $x$ tend vers $a$ }
\textit{ Soit $a \in \ol{X} \cap \R$ }
On dit que $f(x)$ tend vers $+\infty$ lorsque $x$ tend vers $a$ si :    $\forall A \in \R, \exists \eta \gt 0, \forall x \in X, |x - a| \leqslant \eta \implies f(x) \geqslant A$
\textb{ Limite $-\infty$ lorsque $x$ tend vers $a$ }
\textit{ Soit $a \in \ol{X} \cap \R$ }
On dit que $f(x)$ tend vers $-\infty$ lorsque $x$ tend vers $a$ si :    $\forall A \in \R, \exists \eta \gt 0, \forall x \in X, |x - a| \leqslant \eta \implies f(x) \leqslant A$
\textb{ Limite en un point du domaine }
\textit{ Soit $a \in X$ }
Si $f(x)$ admet une limite en $a$, alors cette limite est nécessairement égale à $f(a)$.
\textb{ Limites dans des espaces métriques }
\textit{ Soit $(E, d)$ et $(F, d')$ deux espaces métriques, et $X \subset E$. Soit $f : X \longrightarrow F$. <br/> Soit $a \in \ol{X}$, et $b \in F$. }
Comme dans le cas de $\R$, on peut considérer l'adhérence $\ol{X}$ de $X$ dans $E$. On dit que $f$ admet une limite $b$ en $a$ si : <br/>$\forall \varepsilon \gt 0, \exists \eta \gt 0, \forall x \in X, d(x, a) \lt \eta \implies d'(f(x), b) \lt \varepsilon$

\end{document}